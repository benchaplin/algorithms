\documentclass[11pt]{article}
\usepackage[utf8]{inputenc}
\usepackage{amsmath, amsthm, amssymb}
\usepackage[margin=0.75in]{geometry}
\usepackage{enumerate}

\usepackage{algorithm}
\usepackage{algpseudocode}
\algblock{Input}{EndInput}
\algnotext{EndInput}
\algblock{Output}{EndOutput}
\algnotext{EndOutput}
\newcommand{\Desc}[2]{\State \makebox[2em][l]{#1}#2}
\makeatletter
\newenvironment{breakablealgorithm}
  {% \begin{breakablealgorithm}
   \begin{center}
     \refstepcounter{algorithm}% New algorithm
     \hrule height.8pt depth0pt \kern2pt% \@fs@pre for \@fs@ruled
     \renewcommand{\caption}[2][\relax]{% Make a new \caption
       {\raggedright\textbf{\fname@algorithm~\thealgorithm} ##2\par}%
       \ifx\relax##1\relax % #1 is \relax
         \addcontentsline{loa}{algorithm}{\protect\numberline{\thealgorithm}##2}%
       \else % #1 is not \relax
         \addcontentsline{loa}{algorithm}{\protect\numberline{\thealgorithm}##1}%
       \fi
       \kern2pt\hrule\kern2pt
     }
  }{% \end{breakablealgorithm}
     \kern2pt\hrule\relax% \@fs@post for \@fs@ruled
   \end{center}
  }
\makeatother

\theoremstyle{plain}
\newtheorem{thm}{Theorem}
\newtheorem{cor}{Corollary}
\newtheorem{lem}{Lemma}
\theoremstyle{definition}
\newtheorem*{defn}{Definition}
\newtheorem*{ex}{Example}

\newcommand{\C}{\mathbb{C}}
\newcommand{\R}{\mathbb{R}}
\newcommand{\Q}{\mathbb{Q}}
\newcommand{\Z}{\mathbb{Z}}
\newcommand{\N}{\mathbb{N}}

\title{Design and Analysis of Algorithms: Problem 1-1(b)}
\author{Ben Chaplin}
\date{}

\begin{document}

\maketitle
\tableofcontents

\section{Problem statement}

\textit{Drunken Donuts}, a new wine-and-donuts restaurant chain, wants to build restaurants on many
street corners with the goal of maximizing their total profit.

The street network is described as an undirected graph $G = (V, E)$, where the potential restaurant
sites are the vertices of the graph. Each vertex $u$ has a nonnegative integer value $p(u)$, which describes
the potential \textbf{profit of site $u$}. Two restaurants cannot be built on adjacent vertices (to avoid 
self-competition). You are supposed to design an algorithm that outputs the chosen set $U \subseteq V$ of sites
that maximizes the total profit $\sum_{u \in U} p(u)$.

Suppose that the street network $G$ is acyclic, i.e., a tree.
\bigbreak

\textbf{(b)} Give an efficient algorithm to determine a placement with maximum profit.

\section{Varying-profitable locations}
\subsection{Profit-maximizing algorithm}
 
\begin{breakablealgorithm}
    \caption{DP algorithm for maximizing profit}
    \begin{algorithmic}
        \Input
        \Desc{$G$}{an acyclic graph with weighted vertices describing profit}
        \EndInput
        \Output
        \Desc{$p$}{the maximum profit over non-adjacent vertices}
        \EndOutput
    \end{algorithmic}
    \vspace{3pt}
    \begin{algorithmic}[1]
        \Procedure{maxprofit}{$v$}
            \If{$v$ has no children}
                \State \Return $p(v)$
            \Else
                \State $a \gets p(v) + \sum_{v \in v.grandchildren} \textsc{maxprofit($v$)}$
                \State $b \gets \sum_{v \in v.children} \textsc{maxprofit($v$)}$
                \State \Return max$(a, b)$
            \EndIf
        \EndProcedure
    \end{algorithmic}
\end{breakablealgorithm}

\subsection{Correctness}

\begin{defn}
    Let $v$ be a vertex in a tree, and $H$ be the tree it roots. 
    The \textbf{maxprofit of \textit{v}} is the maximum profit over all subsets $U \subseteq H$, 
    where no two vertices in $U$ are adjacent.
\end{defn}

\begin{proof} (Induction on the height of $G$)\\
    \textbf{Base case 1:} assume $G$ has height 1, and thus only one vertex $v$. 
    Then \textbf{Algorithm 1} will return $p(v)$, which is the maxprofit of $G$.\\
    \noindent\textbf{Base case 2:} assume $G$ has has height 2. Let $h$ be the root of $G$.
    
    \indent\textbf{Case 1:} $p(h) > \sum_{v \in h.children} p(v)$\\
    \indent\indent\textbf{Algorithm 1} will return $p(v)$, as $a > b$.\\
    \indent\textbf{Case 2:} $p(h) \leq \sum_{v \in h.children} p(v)$\\
    \indent\indent\textbf{Algorithm 1} will return $\sum_{v \in h.children} p(v)$, as $b \geq a$.
    \bigbreak

    \noindent\textbf{Inductive hypothesis:} Assume \textbf{Algorithm 1} is correct when run on graphs
    of height $n - 1$ and $n - 2$.
    \bigbreak
    
    \noindent Suppose $|G| = n > 2$ and \textbf{Algorithm 1} is run on $G$. Let $h$ be the root of $G$.
    \bigbreak
    \indent\textbf{Case 1:} suppose $h$ is included in the profit-maximizing subset of
    vertices. 

    Then, \textbf{Algorithm 1} returns the sum of $p(h)$ with recursive calls on the grandchildren of $h$.
    The trees under the grandchildren of $h$ have height $n - 2$, so by our hypothesis, we can be 
    sure that the algorithm returns the maxprofit of each grandchild.  

    Furthermore, no grandchild is adjacent to another, so no member of either's tree is adjacent to 
    a member of another's tree.

    Therefore, \textbf{Algorithm 1} returns the maxprofit of $h$.
    \bigbreak

    \indent\textbf{Case 2:} suppose otherwise.

    Then, \textbf{Algorithm 1} returns the sum $\sum_{v \in h.children} p(v)$. The trees under each 
    child $v$ have height $n - 1$, so by our hypothesis, we can be sure that the algorithm returns 
    the maxprofit under each child. 

    Furthermore, no child is adjacent to another, so no member of either's tree is adjacent to 
    a member of another's tree.

    Therefore, \textbf{Algorithm 1} returns the maxprofit of $h$.
\end{proof}

\subsection{Runtime analysis}

\textbf{Algorithm 1} performs $O(1)$ operations on each vertex in $G$, thus the runtime
is $O(|V|)$.



\end{document}

