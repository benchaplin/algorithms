\documentclass[11pt]{article}
\usepackage[utf8]{inputenc}
\usepackage{amsmath, amsthm, amssymb}
\usepackage[margin=0.75in]{geometry}
\usepackage{enumerate}

\usepackage{algorithm}
\usepackage{algpseudocode}
\algblock{Input}{EndInput}
\algnotext{EndInput}
\algblock{Output}{EndOutput}
\algnotext{EndOutput}
\newcommand{\Desc}[2]{\State \makebox[2em][l]{#1}#2}
\makeatletter
\newenvironment{breakablealgorithm}
  {% \begin{breakablealgorithm}
   \begin{center}
     \refstepcounter{algorithm}% New algorithm
     \hrule height.8pt depth0pt \kern2pt% \@fs@pre for \@fs@ruled
     \renewcommand{\caption}[2][\relax]{% Make a new \caption
       {\raggedright\textbf{\fname@algorithm~\thealgorithm} ##2\par}%
       \ifx\relax##1\relax % #1 is \relax
         \addcontentsline{loa}{algorithm}{\protect\numberline{\thealgorithm}##2}%
       \else % #1 is not \relax
         \addcontentsline{loa}{algorithm}{\protect\numberline{\thealgorithm}##1}%
       \fi
       \kern2pt\hrule\kern2pt
     }
  }{% \end{breakablealgorithm}
     \kern2pt\hrule\relax% \@fs@post for \@fs@ruled
   \end{center}
  }
\makeatother

\theoremstyle{plain}
\newtheorem{thm}{Theorem}
\newtheorem{cor}{Corollary}
\newtheorem{lem}{Lemma}
\theoremstyle{definition}
\newtheorem*{defn}{Definition}
\newtheorem*{ex}{Example}

\newcommand{\C}{\mathbb{C}}
\newcommand{\R}{\mathbb{R}}
\newcommand{\Q}{\mathbb{Q}}
\newcommand{\Z}{\mathbb{Z}}
\newcommand{\N}{\mathbb{N}}

\title{Design and Analysis of Algorithms: Lecture 4}
\author{Ben Chaplin}
\date{}

\begin{document}

\maketitle
\tableofcontents

\section{Background}
\subsection{Operations}

We want to maintain $n$ elements from the set $\{0, 1, \ldots, u - 1\}$, and support the following
operations:

\begin{itemize}
    \item \textsc{insert}($V, x$): insert $x$ into $V$
    \item \textsc{delete}($V, x$): delete $x$ from $V$
    \item \textsc{successor}($V, x$): return the smallest element in $V$ which is larger than $x$
\end{itemize}

\subsection{Problem}

Balanced binary search trees support all three of the above operations in $O(\log n)$ time. The 
goal of \textbf{van Emde Boas trees} is to support operations in $O(\log\log n)$ time.
\bigbreak
Let $n$ and $u$ be defined as they were above. If $u = n^{O(1)}$, then $\log\log u = O(\log\log n)$.

\subsection{Recurrences}

Recall the binary search recurrence:
\begin{align}
    T(k) &= T\left(\frac{k}{2}\right) + O(1)\\
         &= O(\log k)
\end{align}

So we're seeking:
\begin{align}
    T(\log u) &= T\left(\frac{\log u}{2}\right) + O(1)\\
              &= O(\log\log u)
\end{align}

Which can be written:
\begin{align}
    T'(u) &= T'(\sqrt u) + O(1)\\
          &= O(\log\log u)
\end{align}

\section{Van Emde Boas Trees}
\subsection{Data structure}

At their core, \textbf{van Emde Boas Trees} are bit vectors of size $u$ where:
\[
V[i] = 
    \begin{cases}
        0 & \text{if } i \text{ is not present}\\
        1 & \text{if } i \text{ is present}
    \end{cases}
\]

The vector is split into \textbf{clusters} of size $\sqrt u$.

Above each cluster is a tree structure. Consider the elements of the cluster as the
leaves of the tree, and store $(l_0 \vee l_1)$ in the parent node for each two neighboring leaves 
$l_0, l_1$. The \textbf{summary} vector contains the root from each tree above each cluster.

Both clusters and summary vectors store their minimum and maximum values.

\subsection{Definitions}

To make writing algorithms quicker:

\begin{defn}
    Given $x \in \{1, \ldots, u - 1\} \subseteq \N$, \textbf{high}$(x) = \lfloor \frac{x}{\sqrt{u}}\rfloor$
    (this effectively gives us the index of the parent of $x$ in the summary vector).
\end{defn}

\begin{defn}
    Given $x \in \{1, \ldots, u - 1\} \subseteq \N$, \textbf{low}$(x) = x \mod \sqrt{u}$
    (this effectively gives us the index of $x$ in its cluster).
\end{defn}

\begin{defn}
    Given $i, j \in \{1, \ldots, u - 1\} \subseteq \N$, \textbf{index}$(i, j) = i\sqrt{u} + j$
    (index(high($x$), low($x$)) = $x$).
\end{defn}

\subsection{Algorithms}

\begin{breakablealgorithm}
    \begin{algorithmic}[1]
        \Procedure{successor}{$V, x$}
            \If{$|V| \leq 4$}
                \State \Return index of successor \Comment{base case}
            \EndIf
            \If{$x < V.$min}
                \State \Return $V.$min
            \EndIf
            \State $i = \text{high}(x)$
            \If{low$(x) < V.$cluster$[i]$.max} \Comment{successor is in cluster $i$}
                \State $j = \textsc{successor}(V.$cluster$[i],$ low$(x))$
            \Else \Comment{successor is not in cluster $i$, check next in summary}
                \State $i = \textsc{successor}(V.$summary$,$ high$(x))$     
                \State $j = V.$cluster$[i].$min
            \EndIf
            \State \Return index$(i, j)$
        \EndProcedure
    \end{algorithmic}
\end{breakablealgorithm}

\textbf{Runtime analysis:} \textsc{successor} runs in $O(\log\log u)$ time where $|V| = u$.
\begin{itemize}
    \item Each line 7 recursive call reduces the size of $V$ by $\sqrt{u}$, so we will have at most
        $O(\log \log u)$ recursive calls.
    \item If we hit a line 9 recursive call, we will finish in $O(1)$ time, as we need only find the 
        max of $V.$summary, then the max of the cluster it summarizes.
\end{itemize}

\begin{breakablealgorithm}
    \begin{algorithmic}[1]
        \Procedure{insert}{$V, x$}
            \If{$V.$min $= null$}
                \State $V.$min $= V.$max $= x$, \Return 
            \EndIf
            \If{$x < V.$min}
                \State swap($x, V.$min)
            \EndIf
            \If{$x > V.$max}
                \State $V.$max $= x$
            \EndIf
            \If{$V.$cluster[high($x$)].min $= $ null}
                \State $\textsc{insert}(V.$summary, high($x$))
            \EndIf
            \State $\textsc{insert}(V.$cluster[high($x$)], low($x$))
        \EndProcedure
    \end{algorithmic}
\end{breakablealgorithm}

\textbf{Runtime analysis:} \textsc{insert} runs in $O(\log\log u)$ time where $|V| = u$.
\begin{itemize}
    \item Each line 14 recursive call reduces the size of $V$ by $\sqrt{u}$, so we will have at most
        $O(\log \log u)$ recursive calls.

    \item If we hit a line 12 recursive call, we will finish in $O(\log \log u)$ time, as we only need to insert into
        the summaries, of which there are $O(\log \log u)$.
\end{itemize}


















\end{document}

