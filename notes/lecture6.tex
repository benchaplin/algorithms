\documentclass[11pt]{article}
\usepackage[utf8]{inputenc}
\usepackage{amsmath, amsthm, amssymb}
\usepackage[margin=0.75in]{geometry}
\usepackage{enumerate}

\usepackage{algorithm}
\usepackage{algpseudocode}
\algblock{Input}{EndInput}
\algnotext{EndInput}
\algblock{Output}{EndOutput}
\algnotext{EndOutput}
\newcommand{\Desc}[2]{\State \makebox[2em][l]{#1}#2}
\makeatletter
\newenvironment{breakablealgorithm}
  {% \begin{breakablealgorithm}
   \begin{center}
     \refstepcounter{algorithm}% New algorithm
     \hrule height.8pt depth0pt \kern2pt% \@fs@pre for \@fs@ruled
     \renewcommand{\caption}[2][\relax]{% Make a new \caption
       {\raggedright\textbf{\fname@algorithm~\thealgorithm} ##2\par}%
       \ifx\relax##1\relax % #1 is \relax
         \addcontentsline{loa}{algorithm}{\protect\numberline{\thealgorithm}##2}%
       \else % #1 is not \relax
         \addcontentsline{loa}{algorithm}{\protect\numberline{\thealgorithm}##1}%
       \fi
       \kern2pt\hrule\kern2pt
     }
  }{% \end{breakablealgorithm}
     \kern2pt\hrule\relax% \@fs@post for \@fs@ruled
   \end{center}
  }
\makeatother

\theoremstyle{plain}
\newtheorem{thm}{Theorem}
\newtheorem{claim}{Claim}
\newtheorem{cor}{Corollary}
\newtheorem{lem}{Lemma}
\theoremstyle{definition}
\newtheorem*{defn}{Definition}
\newtheorem*{ex}{Example}

\newcommand{\C}{\mathbb{C}}
\newcommand{\R}{\mathbb{R}}
\newcommand{\Q}{\mathbb{Q}}
\newcommand{\Z}{\mathbb{Z}}
\newcommand{\N}{\mathbb{N}}

\title{Design and Analysis of Algorithms: Lecture 6}
\author{Ben Chaplin}
\date{}

\begin{document}

\maketitle
\tableofcontents

\section{Randomization}

A \textbf{randomized} or \textbf{probabilistic} algorithm makes some decisions based on a random 
number generated at runtime. On the same input, a randomized algorithm may:
\begin{itemize}
    \item return different outputs
    \item run differently, but return the same output
\end{itemize}

\begin{defn}
    \textbf{Monte Carlo algorithms} are probably fast, but may not be correct.
\end{defn}

\begin{defn}
    \textbf{Las Vegas algorithms} are certainly fast, but only probably correct.
\end{defn}

\begin{defn}
    \textbf{Atlantic City algorithms} are both probably fast and probably correct.
\end{defn}

\subsection{Matrix Products}

\begin{ex}
    Consider the problem of verifying an $n \times n$ matrix product.
    The standard method of multiplying matricies gives a runtime of $O(n^3)$. Better algorithms 
    have been developed, however they only offer runtimes of around $O(n^{\log 7})$. We seek to 
    use an $O(n^2)$ randomized algorithm where:
    \begin{itemize}
        \item if $A \times B = C$, then $P(\text{returns \textbf{True}}) = 1$
        \item else, if $A \times B \neq C$, then $P(\text{returns \textbf{True}}) \leq \frac{1}{2}$
    \end{itemize}

    Then, running the algorithm on matrices $(A, B, C)$ $k$ times, and getting a result of \textbf{True}
    every time gives a probability of incorrectness:
    $$P(\text{incorrect}) \leq \frac{1}{2^k}$$

    In other words, we can ensure a given level of certainty by running the algorithm a constant number 
    of times. Thus, the overall runtime is still $O(n^2)$. 
\end{ex}

\subsection{Frievold's algorithm}

\begin{breakablealgorithm}
    \caption{Frievold's algorithm}
    \begin{algorithmic}
        \Input
        \Desc{$A, B, C$}\hspace{60pt}{where each is an $n \times n$ matrix}
        \EndInput
        \Output
        \Desc{$\{\text{True, False}\}$}\hspace{60pt}{whether $AB = C$}
        \EndOutput
    \end{algorithmic}
    \vspace{3pt}
    \begin{algorithmic}[1]
        \State $r = $ random binary vector of length $n$
        \If{$A(Br) = Cr$}
            \State \Return \textbf{True}
        \Else
            \State \Return \textbf{False}
        \EndIf
    \end{algorithmic}
\end{breakablealgorithm}

\subsection{Runtime Analysis}

The runtime of Frievold's is:
\begin{itemize}
    \item $O(n)$ to choose a random vector
    \item $O(n^2)$ to compute $Br$, $O(n^2)$ to compute $A(Br)$ and $O(n^2)$ to compute $Cr$
    \item $O(n)$ to compare $A(Br)$ to $Cr$
\end{itemize}

Overall: $O(n^2)$.

\subsection{Correctness}

\begin{claim}
    If $AB = C$, Frievold's always returns \textbf{True}.
\end{claim}

\begin{proof}
    Assume $AB = C$, and $r$ is some vector of length $n$.
    \begin{align*}
        Cr &= (AB)r && \text{by matrix associativity}\\
           &= A(Br)
    \end{align*}
\end{proof}

\begin{claim}
    If $AB \neq C$, $P(\textbf{True}) = \displaystyle\frac{1}{2}$.
\end{claim}

\begin{proof}
    Let $D = AB - C$. Suppose Frievold's generates a random vector $r$ for which it returns 
    \textbf{True}. Then:
    \begin{align*} 
        Dr &= (AB - C)r\\
           &= (AB)r - Cr\\
           &= A(Br) - Cr\\
           &= 0
    \end{align*}
    $D = AB - C \neq 0$, so we can take $i, j \in \{1, \ldots, n\}$ such that
    $D_{ij} = 1$. Let $v$ be the vector:
    \[
        v[k] = 
        \begin{cases}
            0 & \text{if } k \neq j\\
            1 & \text{if } k = j
        \end{cases}
    \]
    Notice that $Dv \neq 0$, because its $j^{th}$ entry will be 1.

    Let $r' = r + v$. Then:
    \begin{align*}
        Dr' &= D(r + v)\\
            &= 0 + Dv\\
            &\neq 0
    \end{align*}
    If $Dr' \neq 0$, $(AB - C)r' = ABr' - Cr' \neq 0$, so Frievold's would return \textbf{False}.
    This defines a one-to-one mapping between vectors $r$ that make Frievold's return \textbf{True},
    and $r'$ that make Frievold's return \textbf{False}.

    Therefore, Frievold's returns \textbf{True} for half of all possible binary vectors of length 
    $n$, and \textbf{False} for the rest. $P(\textbf{True}) = \frac{1}{2}$.
\end{proof}






















\end{document}

