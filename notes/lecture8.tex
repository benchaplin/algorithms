\documentclass[11pt]{article}
\usepackage[utf8]{inputenc}
\usepackage{amsmath, amsthm, amssymb}
\usepackage[margin=0.75in]{geometry}
\usepackage{enumerate}

\usepackage{algorithm}
\usepackage{algpseudocode}
\algblock{Input}{EndInput}
\algnotext{EndInput}
\algblock{Output}{EndOutput}
\algnotext{EndOutput}
\newcommand{\Desc}[2]{\State \makebox[2em][l]{#1}#2}
\makeatletter
\newenvironment{breakablealgorithm}
  {% \begin{breakablealgorithm}
   \begin{center}
     \refstepcounter{algorithm}% New algorithm
     \hrule height.8pt depth0pt \kern2pt% \@fs@pre for \@fs@ruled
     \renewcommand{\caption}[2][\relax]{% Make a new \caption
       {\raggedright\textbf{\fname@algorithm~\thealgorithm} ##2\par}%
       \ifx\relax##1\relax % #1 is \relax
         \addcontentsline{loa}{algorithm}{\protect\numberline{\thealgorithm}##2}%
       \else % #1 is not \relax
         \addcontentsline{loa}{algorithm}{\protect\numberline{\thealgorithm}##1}%
       \fi
       \kern2pt\hrule\kern2pt
     }
  }{% \end{breakablealgorithm}
     \kern2pt\hrule\relax% \@fs@post for \@fs@ruled
   \end{center}
  }
\makeatother

\theoremstyle{plain}
\newtheorem{thm}{Theorem}
\newtheorem{claim}{Claim}
\newtheorem{cor}{Corollary}
\newtheorem{lem}{Lemma}
\theoremstyle{definition}
\newtheorem*{defn}{Definition}
\newtheorem*{ex}{Example}

\newcommand{\C}{\mathbb{C}}
\newcommand{\R}{\mathbb{R}}
\newcommand{\Q}{\mathbb{Q}}
\newcommand{\Z}{\mathbb{Z}}
\newcommand{\N}{\mathbb{N}}

\title{Design and Analysis of Algorithms: Lecture 8}
\author{Ben Chaplin}
\date{}

\begin{document}

\maketitle
\tableofcontents

\section{Problem}
\subsection{Dictionary problem}

The dictionary problem asks for a data structure with the following requirements:
\begin{itemize}
    \item Maintain a dynamic set of items (where each item has a distinct key)
    \item Support:
        \begin{itemize}
            \item \textsc{insert}(item)
            \item \textsc{delete}(item)
            \item \textsc{search}(key)
        \end{itemize}
\end{itemize}

\subsection{Hashing}

``Hashing'' the items' keys in a hash table gives $O(1)$ time per operation. We will use the
following variables with regards to a hash table:
\begin{itemize}
    \item $u:$ number of possible keys 
    \item $n:$ number of items currently in the table
    \item $m:$ size of the table
    \item $h: \{0, 1, \ldots, u-1\} \rightarrow \{0, 1, \ldots, m-1\}$ the hash function
\end{itemize}

With chaining, hashing takes $\Theta(1 + \frac{n}{m})$ time. The proof of this fact assumes 
\textbf{simple uniform hashing}.

\begin{defn}
    A function provides \textbf{simple uniform hashing} when, for two random distinct keys 
    $k_1, k_2$, the probability the function outputs the same hash is $\displaystyle\frac{1}{m}$.
\end{defn}

But what kind of hash functions guarantee this, no matter the universe of keys?

\section{Universal Hashing}

\begin{defn}
    Let $\mathcal{H}$ be a set of hash functions. $\mathcal{H}$ is \textbf{universal} if for any two
    distinct keys $k_1, k_2$, the probability a random hash function $h \in \mathcal{H}$ outputs 
    the same hash is at most $\displaystyle\frac{1}{m}$.
\end{defn}

Note that contrary to the definition of simple uniform hashing, this definition includes a probability 
over \textit{all hash functions}. Simple uniform hashing was defined by a probability over 
\textit{all pairs of distinct keys}.

\begin{thm}
    Let $\mathcal{H}$ be universal. For $n$ arbitrary distinct keys and a random $h \in \mathcal{H}$,
    the expected number of colliding keys is at most $1 + \displaystyle\frac{n}{n}$.
\end{thm}

\begin{proof}
    Take keys $k_1, \ldots, k_n$. Define an ``indicator'' random variable:
    $$I_{i,j} = 
    \begin{cases}
        1 & \text{if $h(k_i) = h(k_j)$}\\
        0 & \text{else}
    \end{cases}$$

    \begin{align*}
        E[\text{number of keys with the same hash as $k_i$}] &= E\left[\sum_{i \neq j} I_{i,j}\right] 
            + I_{i,i}\\
        &= E\left[\sum_{i \neq j} I_{i,j}\right]+ 1\\
        &= \left(\sum_{i \neq j} E[I_{i,j}]\right) + 1\\
        &= \left(\sum_{i \neq j} Pr(I_{i,j} = 1)\right) + 1\\
        &\leq \left(\sum_{i \neq j} \frac{1}{m}\right) + 1 && \text{by universality}\\
        &= \frac{n-1}{m} + 1
    \end{align*}
\end{proof}

\section{Dot product hash family}

\begin{defn}
    Assume $m$ is prime and $u = m^r$ for some $r \in \Z^+$. For each key $k$, define a vector 
    $\bar{k} = \langle k_0, k_1, \ldots, k_{r-1}\rangle$ to be the digits of $k$ in base $m$. 
    The \textbf{dot product hash family} is defined:
    $$\mathcal{H} = \left\{ h_a(k) = (\bar{a} \cdot \bar{k})\mod m \mid a \in \{0, \ldots, u-1\}\right\}$$
\end{defn}

Note here that the hash functions in $\mathcal{H}$ are completely determined by the choice of $a$. 

\begin{thm}
    The dot product hash family is universal.
\end{thm}

\begin{proof}
    Let $k \neq k'$ be keys. Then, some digit of $\bar{k}$ and $\bar{k'}$ differs, say $k_d \neq k'_d$.
\end{proof}




 












\end{document}

