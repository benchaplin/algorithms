\documentclass[11pt]{article}
\usepackage[utf8]{inputenc}
\usepackage{amsmath, amsthm, amssymb}
\usepackage[margin=0.75in]{geometry}

\usepackage{algorithm}
\usepackage{algpseudocode}
\algblock{Input}{EndInput}
\algnotext{EndInput}
\algblock{Output}{EndOutput}
\algnotext{EndOutput}
\newcommand{\Desc}[2]{\State \makebox[2em][l]{#1}#2}

\theoremstyle{plain}
\newtheorem{thm}{Theorem}
\newtheorem{cor}{Corollary}
\newtheorem{lem}{Lemma}
\theoremstyle{definition}
\newtheorem*{defn}{Definition}
\newtheorem*{ex}{Example}

\newcommand{\C}{\mathbb{C}}
\newcommand{\R}{\mathbb{R}}
\newcommand{\Q}{\mathbb{Q}}
\newcommand{\Z}{\mathbb{Z}}
\newcommand{\N}{\mathbb{N}}

\title{Design and Analysis of Algorithms: Lecture 1}
\author{Ben Chaplin}
\date{}

\begin{document}

\maketitle
\tableofcontents

\section{Overview}
\subsection{Course details}

{\bf Course Title}: Design and Analysis of Algorithms\\
{\bf Teacher}: Professors Erik Demaine, Srini Devadas \& Nancy Lynch\\
{\bf School:} MIT\\ 
{\bf Lectures}: https://ocw.mit.edu/courses/electrical-engineering-and-computer-science/6-046j-design-and-\\
analysis-of-algorithms-spring-2015/\\
{\bf Textbook}: {\it Introduction to Algorithms} by Cormen, Leiserson, Rivest \& Stein (3rd ed) 

\subsection{Complexity classes}

\begin{defn}
    {\bf P} is the class of problems solvable in polynomial time.
\end{defn}

\begin{defn}
    {\bf NP} is the class of problems verifiable in polynomial time.
\end{defn}

\begin{ex}
    Given a graph, does there exist a Hamiltonian cycle? This problem is thought not to be in P,
    but is in NP. This is because no algorithm has been found to determine whether a graph contains a Hamiltonian
    cycle in $O(|V|^k)$ time for any $k \in \N$. However, given a graph and a path, it can be verified
    whether the path is a Hamiltonian cycle in polynomial time.
\end{ex}

\begin{defn}
    {\bf NP-complete} problems are problems in NP which are ``as hard'' as any problem in NP.
\end{defn}

\section{Interval Scheduling}
\subsection{Definitions \& Algorithm}

\begin{defn}
    A {\bf request} is an ordered pair of numbers $(s, f) \in \N \times \N$ representing ``start'' and ``finish'' times.
\end{defn}

\begin{defn}
    Two requests $r_i = (s_i, f_i)$ and $r_j = (s_j, f_j)$ are {\bf compatible} if $f_i \leq s_j$ and
    $f_j \leq s_i$ (they don't overlap). A set of requests is {\bf compatible} if every pair of requests is compatible.
\end{defn}

\begin{defn}
    Given a set of requests $R$, the \textbf{optimal schedule of $R$} is the largest-cardinality compatible 
    subset of $R$.
\end{defn}

\begin{algorithm}
    \caption{Greedy Algorithm for Interval Scheduling}
    \begin{algorithmic}
        \Input
        \Desc{$R$}{a list of requests, ordered by finish time}
        \EndInput
        \Output
        \Desc{$C$}{the optimal schedule of $R$}
        \EndOutput
    \end{algorithmic}
    \vspace{3pt}
    \begin{algorithmic}[1]
        \State $R' \gets R$
        \State $O \gets \{\}$
        \While{$R'$ is not empty}
            \State $r \gets$ the request in $R'$ with the earliest finish time
            \State Add $r$ to $O$
            \State Remove all requests from $R'$ which are not compatible with $r$
        \EndWhile
        \State \Return $O$
    \end{algorithmic}
\end{algorithm}

\subsection{Correctness}

First, we'll prove the correctness of this algorithm.

\begin{thm}
    Given a list of requests $R$ (ordered by finish time), \textbf{Algorithm 1} produces an optimal schedule of $R$.
\end{thm}

\begin{proof}
    (Induction on $k^* = $ the size of the optimal schedule)\\
    \indent\textbf{Base case:}
    Suppose $k^* = 1$. Then {\bf Algorithm 1} returned one request $r$, where $r$ has the earliest 
    finish time, and the rest of the requests are incompatible with $r$. Therefore, there can only be
    one compatible request, so {\bf Algorithm 1} succeeded.
    \bigbreak

    \textbf{Inductive hypothesis:}
    If a list of requests $R$ has an optimal schedule of size $k^*$, then {\bf Algorithm 1} will produce
    a schedule of size $k^*$.
    \bigbreak

    Let $S$ be a list of requests which has an optimal schedule $O^*$ of size $(k^* + 1)$:
    $$O^*[1, \ldots, k^* + 1] = (s_{j_1}, f_{j_1}), \ldots, (s_{j_{k^* + 1}}, f_{j_{k^* + 1}})$$

    Suppose, when \textbf{Algorithm 1} is run on $S$, it produces the schedule $O$ of size $k$:
    $$O[1, \ldots, k] = (s_{i_1}, f_{i_1}), \ldots, (s_{i_k}, f_{i_k})$$

    Note that $f_{i_1} \leq f_{j_1}$, because by definition, the algorithm chooses the request with the earliest 
    finish time. Then, we can define a new schedule by taking $O^*$, and replacing $O^*[1]$ with $O[1]$:
    $$O^{**}[1, \ldots, k^* + 1] = (s_{i_1}, f_{i_1}), (s_{j_2}, f_{j_2}), \ldots, (s_{j_{k^* + 1}}, f_{j_{k^* + 1}})$$

    Note that $O^{**}$ is also optimal, as it has length $(k^* + 1)$. We define a new list $S'$:
    $$S' = \{(s_i, f_i) \mid (s_i, f_i) \in S \text{ and } s_i \geq f_{i_1}\}$$

    In other words, $S'$ is the list of intervals in $S$ which are compatible with $(s_{i_1}, f_{i_1})$.

    Because $O^{**}$ is optimal for $S$, $O^{**}[2, \ldots, k^* + 1]$ is optimal for $S'$. By the 
    \textbf{Inductive hypothesis}, \textbf{Algorithm 1} should produce a schedule of $k^*$ requests when 
    run on $S'$. So, when run on $S$, \textbf{Algorithm 1} will:
    \begin{enumerate}
        \item choose $(s_{i_1}, f_{i_1})$ (we've stated this explicitly),
        \item then choose another $k^*$ requests (as now, the algorithm is effectively running on $S'$).
    \end{enumerate}
    So \textbf{Algorithm 1} will return the optimal schedule of size $(k^* + 1)$.
\end{proof}

\subsection{Runtime}

\textbf{Algorithm 1} runs in $O(n^2)$ time, as:
\begin{enumerate}
    \item it iterates $O(n)$ times (no request can be processed more than once),
    \item and for each selected request, every subsequent request is checked, resulting in another $O(n)$.
\end{enumerate}



















\end{document}

