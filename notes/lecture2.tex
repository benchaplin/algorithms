\documentclass[11pt]{article}
\usepackage[utf8]{inputenc}
\usepackage{amsmath, amsthm, amssymb}
\usepackage[margin=0.75in]{geometry}

\usepackage{algorithm}
\usepackage{algpseudocode}
\algblock{Input}{EndInput}
\algnotext{EndInput}
\algblock{Output}{EndOutput}
\algnotext{EndOutput}
\newcommand{\Desc}[2]{\State \makebox[2em][l]{#1}#2}
\makeatletter
\newenvironment{breakablealgorithm}
  {% \begin{breakablealgorithm}
   \begin{center}
     \refstepcounter{algorithm}% New algorithm
     \hrule height.8pt depth0pt \kern2pt% \@fs@pre for \@fs@ruled
     \renewcommand{\caption}[2][\relax]{% Make a new \caption
       {\raggedright\textbf{\fname@algorithm~\thealgorithm} ##2\par}%
       \ifx\relax##1\relax % #1 is \relax
         \addcontentsline{loa}{algorithm}{\protect\numberline{\thealgorithm}##2}%
       \else % #1 is not \relax
         \addcontentsline{loa}{algorithm}{\protect\numberline{\thealgorithm}##1}%
       \fi
       \kern2pt\hrule\kern2pt
     }
  }{% \end{breakablealgorithm}
     \kern2pt\hrule\relax% \@fs@post for \@fs@ruled
   \end{center}
  }
\makeatother

\theoremstyle{plain}
\newtheorem{thm}{Theorem}
\newtheorem{cor}{Corollary}
\newtheorem{lem}{Lemma}
\theoremstyle{definition}
\newtheorem*{defn}{Definition}
\newtheorem*{ex}{Example}

\newcommand{\C}{\mathbb{C}}
\newcommand{\R}{\mathbb{R}}
\newcommand{\Q}{\mathbb{Q}}
\newcommand{\Z}{\mathbb{Z}}
\newcommand{\N}{\mathbb{N}}

\title{Design and Analysis of Algorithms: Lecture 2}
\author{Ben Chaplin}
\date{}

\begin{document}

\maketitle
\tableofcontents

\section{Divide \& Conquer}
\subsection{Paradigm}

Given a problem of size $n$, the general strategy of divide \& conquer can be summarized:
\begin{enumerate}
    \item Divide the problem into $a$ ($a \geq 1)$ subproblems of size $\frac{n}{b}$ ($b > 1$).
    \item Recursively solve subproblems.
    \item Combine solutions of the subproblems into one overall solution.
\end{enumerate}

\section{The Convex Hull problem}
\subsection{Definitions}

\begin{defn}
    Given a set of points $S$ on the real plane, \textbf{convex hull of S} ($CH(S)$) is the smallest 
    convex polygon containing all of the points in $S$.
\end{defn}

We'll represent a convex hull as a sequence of points in clockwise order.

\subsection{Brute-force algorithm}

\begin{breakablealgorithm}
    \caption{Brute-force algorithm for the Convex Hull problem}
    \begin{algorithmic}
        \Input
        \Desc{$S$}{a set of points in the real plane, where:
        \begin{enumerate}
            \setlength{\itemindent}{.5in}
            \item no two points have the same $x$-coordinate,
            \item no two points have the same $y$-coordinate,
            \item and no three points are colinear
        \end{enumerate}}
        \EndInput
        \Output
        \Desc{$CH(S)$}{\hspace{16pt} the convex hull of $S$ (as a sequence of points in clockwise order)}
        \EndOutput
    \end{algorithmic}
    \vspace{3pt}
    \begin{algorithmic}[1]
        \State $C \gets \{\}$
        \For{each pair of points $p, q \in S$}
            \State Draw the line which holds both points
            \If{all remaining points lie on one side of the line}
                \State Add the points $p, q$ to $C$
            \EndIf
        \EndFor
        \State \Return the elements of $C$, clockwise-ordered
    \end{algorithmic}
\end{breakablealgorithm}

\subsection{Brute-force runtime}

The brute-force algorithm:
\begin{enumerate}
    \item iterates $O(n^2)$ times (line 2)
    \item during each, iterates $O(n)$ times (line 4)
\end{enumerate}

The sorting of the elements before returning runs in linear time and occurs only once, 
therefore, \textbf{Algorithm 1} runs in $O(n^3)$ time.

\subsection{Divide \& conquer algorithm}

\begin{breakablealgorithm}
    \caption{Divide \& conquer algorithm for the Convex Hull problem}
    \begin{algorithmic}
        \Input
        \Desc{$S$}{a set of points in the real plane, where:
        \begin{enumerate}
            \setlength{\itemindent}{.5in}
            \item no two points have the same $x$-coordinate,
            \item no two points have the same $y$-coordinate,
            \item and no three points are colinear
        \end{enumerate}}
        \EndInput
        \Output
        \Desc{$CH(S)$}{\hspace{16pt} the convex hull of $S$ (as a sequence of points in clockwise order)}
        \EndOutput
    \end{algorithmic}
    \vspace{3pt}
    \begin{algorithmic}[1]
        \State b
    \end{algorithmic}
\end{breakablealgorithm}












\end{document}

