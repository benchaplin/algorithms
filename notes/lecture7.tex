\documentclass[11pt]{article}
\usepackage[utf8]{inputenc}
\usepackage{amsmath, amsthm, amssymb}
\usepackage[margin=0.75in]{geometry}
\usepackage{enumerate}

\usepackage{algorithm}
\usepackage{algpseudocode}
\algblock{Input}{EndInput}
\algnotext{EndInput}
\algblock{Output}{EndOutput}
\algnotext{EndOutput}
\newcommand{\Desc}[2]{\State \makebox[2em][l]{#1}#2}
\makeatletter
\newenvironment{breakablealgorithm}
  {% \begin{breakablealgorithm}
   \begin{center}
     \refstepcounter{algorithm}% New algorithm
     \hrule height.8pt depth0pt \kern2pt% \@fs@pre for \@fs@ruled
     \renewcommand{\caption}[2][\relax]{% Make a new \caption
       {\raggedright\textbf{\fname@algorithm~\thealgorithm} ##2\par}%
       \ifx\relax##1\relax % #1 is \relax
         \addcontentsline{loa}{algorithm}{\protect\numberline{\thealgorithm}##2}%
       \else % #1 is not \relax
         \addcontentsline{loa}{algorithm}{\protect\numberline{\thealgorithm}##1}%
       \fi
       \kern2pt\hrule\kern2pt
     }
  }{% \end{breakablealgorithm}
     \kern2pt\hrule\relax% \@fs@post for \@fs@ruled
   \end{center}
  }
\makeatother

\theoremstyle{plain}
\newtheorem{thm}{Theorem}
\newtheorem{claim}{Claim}
\newtheorem{cor}{Corollary}
\newtheorem{lem}{Lemma}
\theoremstyle{definition}
\newtheorem*{defn}{Definition}
\newtheorem*{ex}{Example}

\newcommand{\C}{\mathbb{C}}
\newcommand{\R}{\mathbb{R}}
\newcommand{\Q}{\mathbb{Q}}
\newcommand{\Z}{\mathbb{Z}}
\newcommand{\N}{\mathbb{N}}

\title{Design and Analysis of Algorithms: Lecture 7}
\author{Ben Chaplin}
\date{}

\begin{document}

\maketitle
\tableofcontents

\section{Skip Lists}
\subsection{Intuition}

We can understand skip lists as offering the same benefits of an express train. Consider a subway
with stops:
\begin{alignat*}{3}
    14 \leftrightarrow 23 \leftrightarrow 34 \leftrightarrow 42 \leftrightarrow 59 \leftrightarrow 66 \leftrightarrow 72 \leftrightarrow 79
\end{alignat*}

We can imagine two trains, one stopping at each station, and the other skipping stops:
\begin{alignat*}{4}
    &14 \leftrightarrow &&34 \leftrightarrow &&59 \leftrightarrow &&72\\
    &14 \leftrightarrow 23 \leftrightarrow &&34 \leftrightarrow 42 \leftrightarrow &&59 \leftrightarrow 66 \leftrightarrow &&72 \leftrightarrow 79
\end{alignat*}

Then, if we start at station 14 and want to get to station 66, we can save some stops by taking the 
``express'' $14 \rightarrow 34 \rightarrow 59$, switch down to the ``local'', then finish $59 \rightarrow 66$.

An even better system might have an ``double-express'' train:
\begin{alignat*}{4}
    &14 \leftrightarrow && &&59\\
    &14 \leftrightarrow &&34 \leftrightarrow &&59 \leftrightarrow &&72\\
    &14 \leftrightarrow 23 \leftrightarrow &&34 \leftrightarrow 42 \leftrightarrow &&59 \leftrightarrow 66 \leftrightarrow &&72 \leftrightarrow 79
\end{alignat*}

...and so on.
 
\subsection{Data structure}

Skip lists maintain a dynamic set of $n$ elements with $O(\log n)$ time per operation expectation, 
\textbf{with high probability}.

\begin{defn}
    A parametrized event $E_\alpha$ occurs \textbf{with high probability} if, for any $\alpha \geq 1$, 
    there is an appropriate choice of constants for which $E_\alpha$ occurs with probability at least
    $1 - \displaystyle\frac{c_\alpha}{n^\alpha}$.
\end{defn}

It keeps all elements in a linked list at ``layer 0'' or $L_0$. It maintains layers of linked lists 
above, each of which may or may not contain one of the elements below. If an element is in $L_i$, 
then it is also in $L_{i - 1}, L_{i - 2}, \ldots, L_0$. Each node in one of the linked lists stores:
\begin{itemize}
    \item a value
    \item a pointer to the next element of the list in the same layer, if not at the end
    \item a pointer to the previous element of the list in the same layer, if not at the beginning
    \item a pointer to the same element of the list in the above layer, if it exists
    \item a pointer to the same element of the list in the below layer, if not at $L_0$
\end{itemize}

\subsection{Algorithms}

\begin{breakablealgorithm}
    \caption{\textsc{search}}
    \begin{algorithmic}
        \Input
        \Desc{$S, x$}\hspace{10pt}{the skip list, and value for which to search}
        \EndInput
    \end{algorithmic}
    \begin{algorithmic}[1]
        \State $e = S.head$
        \While{\textbf{true}}
            \While{$e.next \neq null$ and $e.next.value < x$}
                \State $e = e.next$
            \EndWhile
            \If{$e.value = x$}
                \State \Return \textbf{true}
            \EndIf
            \If{$e.down \neq null$}
                \State $e = e.down$
            \Else
                \State \Return \textbf{false}
            \EndIf
        \EndWhile
    \end{algorithmic}
\end{breakablealgorithm}

\begin{breakablealgorithm}
    \caption{\textsc{insert}}
    \begin{algorithmic}
        \Input
        \Desc{$S, x$}\hspace{10pt}{the skip list, and value to insert}
        \EndInput
    \end{algorithmic}
    \begin{algorithmic}[1]
        \State $e = \textsc{search-traverse}(S, x)$ 
        \Comment{same as \textsc{search}, but stop before returning \textbf{false} and just return $e$}
        \State $f = \text{new Node}(x, prev: e.prev, next: e.next)$
        \State $e.next.prev = f$
        \State $e.next = f$
        \State $r = \textsc{flip-coin}()$
        \While{$r = $ heads}
            \State $e.up = \text{new Node}(x, prev, next)$
            \State $e = e.up$
            \State $r = \textsc{flip-coin}()$
        \EndWhile
    \end{algorithmic}
\end{breakablealgorithm}

\begin{breakablealgorithm}
    \caption{\textsc{delete}}
    \begin{algorithmic}
        \Input
        \Desc{$S, x$}\hspace{10pt}{the skip list, and value to delete}
        \EndInput
    \end{algorithmic}
    \begin{algorithmic}[1]
        \State $e = \textsc{search-traverse}(S, x)$ 
        \Comment{same as \textsc{search}, but stop before returning \textbf{false} and just return $e$}
        \State $e.prev\:?\:e.prev.next = e.next$
        \State $e.next\:?\:e.next.prev = e.prev$
        \While{$e.up \neq null$}
            \State $e = e.up$
            \State $e.prev\:?\:e.prev.next = e.next$
            \State $e.next\:?\:e.next.prev = e.prev$
        \EndWhile
    \end{algorithmic}
\end{breakablealgorithm}

\subsection{Analysis}

\begin{lem}
    The number of layers after inserting $n$ elements into a skip list is $O(\log n)$, with high
    probability. 
\end{lem}

\begin{proof}
    \begin{align*}
         Pr(\text{number of layers }\neq c \cdot \log n) &= Pr(\text{number of layers }> c \cdot \log n
         \: \forall c \in \R^+)\\
                                                         &= Pr(\text{some insert flipped heads }
                                                         > c \cdot \log n \text{ times})\\
                                                         &\leq n \cdot Pr(\text{a particular insert 
                                                             flipped heads }
                                                         > c \cdot \log n \text{ times})\\
                                                         &= n \left(\frac{1}{2}\right)^{c \log n}\\
                                                         &= n \cdot \frac{1}{n^c}\\
                                                         &= \frac{1}{n^{c-1}}
    \end{align*}
    Therefore, the probability that the number of layers in the skip list after $n$ inserts is
    $O(\log n)$ is $1 - \displaystyle\frac{1}{n^{c-1}}$.
\end{proof}

\begin{thm}
    Searching for an element (\textbf{Algorithm 1}) in an $n$-element skip list costs $O(\log n)$ 
    time, with high probability.
\end{thm}

\begin{proof}
    Consider a search traversal backwards (from $L_0$ up/left). For each node: 
    \begin{itemize}
        \item If there is a node ``up'', we flipped heads
        \item If there is no node ``up'', we flipped tails 
    \end{itemize}
    \begin{align*}
        \text{The number of ``up'' moves } &< \text{ the number of levels}\\
                                           &\leq c \log n \quad\text{w.h.p.} && \text{by \textbf{Lemma 1}}
    \end{align*}
    Therefore, w.h.p., the number of moves is at most the number of coin flips until we get $c \log n$
    heads. Let $a$ be arbitrarily large. Say we flip the coin $a\cdot c \log n$ times.
    \begin{align*}
        Pr(\text{exactly $c \log n$ heads}) &= \binom{a \cdot c\log n}{c \log n} 
        \left(\frac{1}{2}\right)^{c\log n} \left(\frac{1}{2}\right)^{(a-1)c\log n}\\
        Pr(\text{at most $c \log n$ heads}) &\leq \binom{a \cdot c\log n}{c \log n} 
        \left(\frac{1}{2}\right)^{(a-1)c\log n}\\
                                            &\leq \frac{(e \cdot a)^{c\log n}}{2^{(a-1)c\log n}}
            && \text{Recall: } \displaystyle\binom{y}{x} \leq \left(e\displaystyle\frac{y}{x}\right)^x\\
            &= \frac{2^{\log (e \cdot a)c \log n}}{2^{(a-1)c\log n}}\\
            &= 2^{\log (e \cdot a) - (a - 1)}\\
            &= \frac{1}{2^{a-1-log(e \cdot a)}}
    \end{align*}
    Then the probability that we get more than $c \log n$ heads is $1 - \displaystyle\frac{1}{n^\alpha}$
    for $\alpha = a - 1 - \log(e \cdot a)$. Therefore, the number of coin flips we need to make until 
    we get $c \log n$ heads is $O(\log n)$, w.h.p.

    Remember that the number of moves in a search is at most the number of coin flips until we get 
    $c\log n$ heads. So search costs $O(\log n)$ time, with high probability.
\end{proof}

















\end{document}

