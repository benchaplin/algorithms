\documentclass[11pt]{article}
\usepackage[utf8]{inputenc}
\usepackage{amsmath, amsthm, amssymb}
\usepackage[margin=0.75in]{geometry}
\usepackage{enumerate}

\usepackage{algorithm}
\usepackage{algpseudocode}
\algblock{Input}{EndInput}
\algnotext{EndInput}
\algblock{Output}{EndOutput}
\algnotext{EndOutput}
\newcommand{\Desc}[2]{\State \makebox[2em][l]{#1}#2}
\makeatletter
\newenvironment{breakablealgorithm}
  {% \begin{breakablealgorithm}
   \begin{center}
     \refstepcounter{algorithm}% New algorithm
     \hrule height.8pt depth0pt \kern2pt% \@fs@pre for \@fs@ruled
     \renewcommand{\caption}[2][\relax]{% Make a new \caption
       {\raggedright\textbf{\fname@algorithm~\thealgorithm} ##2\par}%
       \ifx\relax##1\relax % #1 is \relax
         \addcontentsline{loa}{algorithm}{\protect\numberline{\thealgorithm}##2}%
       \else % #1 is not \relax
         \addcontentsline{loa}{algorithm}{\protect\numberline{\thealgorithm}##1}%
       \fi
       \kern2pt\hrule\kern2pt
     }
  }{% \end{breakablealgorithm}
     \kern2pt\hrule\relax% \@fs@post for \@fs@ruled
   \end{center}
  }
\makeatother

\theoremstyle{plain}
\newtheorem{thm}{Theorem}
\newtheorem{cor}{Corollary}
\newtheorem{lem}{Lemma}
\theoremstyle{definition}
\newtheorem*{defn}{Definition}
\newtheorem*{ex}{Example}

\newcommand{\C}{\mathbb{C}}
\newcommand{\R}{\mathbb{R}}
\newcommand{\Q}{\mathbb{Q}}
\newcommand{\Z}{\mathbb{Z}}
\newcommand{\N}{\mathbb{N}}

\title{Design and Analysis of Algorithms: Lecture 3}
\author{Ben Chaplin}
\date{}

\begin{document}

\maketitle
\tableofcontents

\section{Polynomials}

\begin{defn}
    A \textbf{polynomial of degree $n$} is a function of the form $A(x) = a_0 + a_1x + \ldots + 
    a_{n - 1}x^{n-1}$, with the \textbf{coefficient vector} $\langle a_0, a_1, \ldots, a_{n - 1}
    \rangle$.
\end{defn}

\subsection{Operations}

\begin{itemize}
    \item \textbf{Evaluation:} $f: A(x), x_0 \mapsto A(x_0)$\\
        A naive algorithm would take $O(n^2)$ time, but if one saves the values of $x_0^k$
        (\textbf{Horner's rule}), evaluation takes $O(n)$.

    \item \textbf{Addition:} $f: A(x), B(x) \mapsto A(x) + B(x)$\\
        Takes $O(n)$ time.

    \item \textbf{Multiplication:} $f: A(x), B(x) \mapsto A(x) B(x)$\\
        If we use coefficient form, it seems the best that can be done is $O(n^2)$.
\end{itemize}

Our problem for today: can we do polynomial multiplication faster than $O(n^2)$. 

\subsection{Representations}

There are more ways to represent a polynomial $A(x) = a_0 + a_1x + \ldots a_{n-1}x^{n-1}$.

\begin{itemize}
    \item \textbf{Coefficient vector:} $\langle a_0, a_1, ldots, a_{n-1} \rangle$ (what we're used to).
    \item \textbf{Roots:} $r_0, r_1, \ldots, r_{n - 1}$ where $A(x) = c(x - r_0)(x - r_1)\cdots
        (x - r_{n-1})$ for some constant $c$ (by the Fundamental Theorem of Algebra, $n$ roots 
        uniquely determine a polynomial).
    \item \textbf{Samples:} $(x_k, y_k)$ for $k = 0, 1, \ldots, n - 1$, where $A(x_k) = y_k$, and
        the $x_k$'s are unique ($n$ unique samples also uniquely determine a polynomial, by the FTA).
\end{itemize}

Differing the representation of polynomials results in different complexities of the operations 
discussed above:

\begin{center}
    \begin{tabular}{ c c c c }
        & \textbf{Coefficient vector} & \textbf{Roots} & \textbf{Samples} \\ 
        \textbf{Evaluation} & $O(n)$ & $O(n)$ & $O(n^2)$\\
        \textbf{Addition} & $O(n)$ & $\infty$ & $O(n)$\\
        \textbf{Multiplication} & $O(n^2)$ & $O(n)$ & $O(n)$\\
    \end{tabular}
\end{center}

\section{Fast Fourier Transform}

So how can we do polynomial multiplication in $\leq O(n^2)$ time?

The idea is to convert polynomials from coefficient form to sample form. Then, multiplying them takes 
$O(n)$ time, as seen above. Then, we will convert the product back to coefficient form. Thus, what
we need is:
\begin{itemize}
    \item A way to convert polynomials: \textbf{coefficients} $\rightarrow$ \textbf{samples} in 
        $\leq O(n\log n)$ time.
    \item A way to convert polynomials: \textbf{samples} $\rightarrow$ \textbf{coefficients} in 
        $\leq O(n\log n)$ time.
\end{itemize}

\textbf{Fast Fourier Transform} is the name of the algorithm we will use to do these conversions.

\subsection{Naive approach}

\textbf{Coefficients} $\rightarrow$ \textbf{Samples}:

Let $\langle a_0, a_1, \ldots, a_{n-1} \rangle$ be the coefficient vectors of a polynomial $A$.
Select some $X = \langle x_0, x_1, \ldots, x_{n-1} \rangle$ to define the sample. Then, we want to calculate
$Y = \langle y_0, y_1, \ldots, y_{n-1} \rangle$, where $A(x_k) = y_k$. This can be represented as 
evaluating the following equation:

$$
V = \begin{pmatrix}
    1 & x_0 & x_0^2 & \dots & x_0^{n-1}\\
    1 & x_1 & x_1^2 & \dots & x_1^{n-1}\\
    \vdots\\
    1 & x_{n-1} & x_{n-1}^2 & \ldots & x_{n-1}^{n-1}
\end{pmatrix},
A = \begin{pmatrix}
    a_0\\
    a_1\\
    \vdots\\
    a_{n-1}
\end{pmatrix},
Y = \begin{pmatrix}
    y_0\\
    y_1\\
    \vdots\\
    y_{n-1}
\end{pmatrix}
$$

$$ V \cdot A = Y $$

However, this takes $O(n^2)$ time.
\bigbreak
\noindent\textbf{Samples} $\rightarrow$ \textbf{Coefficients}:

Now suppose we are given $X$ and $Y$, and asked to determine $A$:

$$V^{-1} \cdot Y = A$$

This also takes $O(n^2)$ time.

\subsection{Collapsing sets}

We will use divide \& conquer to evaluate the polynomial. However, if we continue using any values
for $X$, we will face the following problem -- our recursion will look like:

$$T(n, |X|) = 2T(\frac{n}{2}, |X|) + O(n + |X|)$$

Note, while $n$ shrinks by half as we recurse, $|X|$ stays the same. This needs to change if we are
to beat $O(n^2)$.

\begin{defn}
    A set $X$ is \textbf{collapsing} if $|X^2| = \displaystyle\frac{|X|}{2}$ and $X^2$ is collapsing, 
    or $|X| = 1$.
\end{defn}

\begin{ex}
    Let us build some collapsing sets:
    \begin{itemize}
        \item Let $X = \{ 1 \}$, then $X$ collapses by the base case.
        \item Let $X = \{ -1, 1 \}$, then $X^2 = \{ 1 \}$, which collapses.
        \item Let $X = \{ -i, i, -1, 1 \}$, then $X^2 = \{ -1, 1 \}$, which collapses.
        \item $\cdots$
    \end{itemize}

    We see by this example that the $(2^n)^{th}$ roots of unity collapse for any $n \in \N$.
\end{ex}

\begin{defn}
    The \textbf{Discrete Fourier Transform} is the matrix-vector product $V \cdot A$ for 
    $x_k = e^{ik\tau / n}$ (where $\tau = 2\pi$).
\end{defn}

So what we want to compute is the \textbf{Discrete Fourier Transform}, as $e^{ik\tau/n}$ are the 
$n^{th}$ roots of unity. This will give us our sample, and we can comupte it in $O(n\log n)$ time.

\subsection{Algorithm}

\begin{breakablealgorithm}
    \caption{Divide \& conquer algorithm for FFT}
    \begin{algorithmic}
        \Input
        \Desc{$A$}{polynomial represented as coefficient vectors}
        \EndInput
        \Output
        \Desc{$(X, Y)$}\hspace{20pt}{a valid sample of $A$}
        \EndOutput
    \end{algorithmic}
    \vspace{3pt}
    \begin{algorithmic}[1]
        \State Pad $A$ so that $|A| = 2^m$ for the smallest $m \in \N$ such that $2^m \geq |A|$.
        \State $n \gets 2^m$
        \State $X \gets \{ e^{ik\tau/n} \mid k = 0, 1, \ldots, n - 1\}$
        \State $Y \gets \textsc{dc}(A, X)$.values
        \State \Return $(X, Y)$
        \Procedure{dc}{$A, X$}
            \State $A_e = \sum_{k=0}^{n/2} a_{2k}x^k$
            \State $A_o = \sum_{k=0}^{n/2} a_{2k+1}x^k$
            \If{$|A| = 2$}
                \For{$x \in X$} \Comment{$X = \{-1, 1\}$ here}
                    \State $M[x] = \textsc{merge}(A_e, A_o, x, \text{null})$
                \EndFor
                \State \Return $M$
            \EndIf
            \State $M_e = \textsc{dc}(A_e, X^2)$
            \State $M_o = \textsc{dc}(A_o, X^2)$
            \State \Return $\textsc{merge}(A_e, M_e, A_o, M_o, x)$
        \EndProcedure
        \Procedure{merge}{$A_e, M_e, A_o, M_o, x$}
            \State \Return $A_e(x^2) + x(A_o(x^2))$ \Comment{using $M_e$ and $M_o$ to calculate}
        \EndProcedure
    \end{algorithmic}
\end{breakablealgorithm}

\subsection{Runtime}

\textbf{Algorithm 1} runs in $O(n\log n)$ time, as each recursive call takes $O(\log n)$ time, and 
we make $O(n)$ recursive calls.

\subsection{Complete Algorithm}

Note that \textbf{Algorithm 1} simply transforms a polynomial from coefficient form to sample form.
The entire FFT algorithm for polynomial multiplication is as follows:

\begin{enumerate}
    \item Run \textbf{Algorithm 1} on both polynomials -- $O(n\log n)$
    \item Multiply the polynomials in sample form -- $O(n)$
    \item Using the inverse DFT, run the same \textbf{Algorithm 1} to convert back to coefficient 
        form -- $O(n\log n)$
\end{enumerate}

Therefore, the complete FFT algorithm also runs in $O(n\log n)$ time.





















\end{document}

